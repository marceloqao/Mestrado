1.1.3 Random Number Generators(RNG)

The first type of sequence generator is a random number generator (RNG). An RNG uses a nondeterministic source (i.e., the entropy source), along with some processing function (i.e., the entropy distillation process) to produce randomness. The use of a distillation process is needed to overcome any weakness in the entropy source that results in the production of non-random numbers (e.g., the occurrence of long strings of zeros or ones). The entropy source typically consists of some physical quantity, such as the noise in an electrical circuit, the timing of user processes (e.g., key strokes or mouse movements), or the quantum effects in a semiconductor. Various combinations of these inputs may be used. The outputs of an RNG may be used directly as a random number or may be fed into a pseudorandom number generator (PRNG). To be used directly (i.e., without further processing), the output of any RNG needs to satisfy strict randomness criteria as measured by statistical tests in order to determine that the physical sources of the RNG inputs appear random. For example, a physical source such as electronic noise may contain a superposition of regular structures, such as waves or other periodic phenomena, which may appear to be random, yet are determined to be non-random using statistical tests. For cryptographic purposes, the output of RNGs needs to be unpredictable. However, some physical sources (e.g., date/time vectors) are quite predictable. These problems may be mitigated by combining outputs from different types of sources to use as the inputs for an RNG. However, the resulting outputs from the RNG may still be deficient when evaluated by statistical tests. In addition, the production of high-quality random numbers may be too time consuming, making such production undesirable when a large quantity of random numbers is needed. To produce large quantities of random numbers, pseudorandom number generators may be preferable. 

1.1.4 Pseudorandom Number Generators (PRNGs)
The second generator type is a pseudorandom number generator (PRNG). A PRNG uses one or more inputs and generates multiple “pseudorandom” numbers. Inputs to PRNGs are called seeds. In contexts in which unpredictability is needed, the seed itself must be random and unpredictable. Hence, by default, a PRNG should obtain its seeds from the outputs of an RNG; i.e., a PRNG requires a RNG as a companion. The outputs of a PRNG are typically deterministic functions of the seed; i.e., all true randomness is confined to seed generation. The deterministic nature of the process leads to the term “pseudorandom.” Since each element of a pseudorandom sequence is reproducible from its seed, only the seed needs to be saved if reproduction or validation of the pseudorandom sequence is required. Ironically, pseudorandom numbers often appear to be more random than random numbers obtained from physical sources. If a pseudorandom sequence is properly constructed, each value in the sequence is produced from the previous value via transformations that appear to introduce additional randomness. A series of such transformations can eliminate statistical auto-correlations between input and output. Thus, the outputs of a PRNG may have better statistical properties and be produced faster than an RNG.  

1.1.5 Testing
Various statistical tests can be applied to a sequence to attempt to compare and evaluate the sequence to a truly random sequence. Randomness is a probabilistic property; that is, the properties of a random sequence can be characterized and described in terms of probability. The likely outcome of statistical tests, when applied to a truly random sequence, is known a priori and can be described in probabilistic terms. There are an infinite number of possible statistical tests, each assessing the presence or absence of a “pattern” which, if detected, would indicate that the sequence is nonrandom. Because there are so many tests for judging whether a sequence is random or not, no specific finite set of tests is deemed “complete.” In addition, the results of statistical testing must be interpreted with some care and caution to avoid incorrect conclusions about a specific generator (see Section 4). A statistical test is formulated to test a specific null hypothesis (H0). For the purpose of this document, the null hypothesis under test is that the sequence being tested is random. Associated with this null hypothesis is the alternative hypothesis (Ha), which, for this document, is that the sequence is not random. For each applied test, a decision or conclusion is derived that accepts or rejects the null hypothesis, i.e., whether the generator is (or is not) producing random values, based on the sequence that was produced. For each test, a relevant randomness statistic must be chosen and used to determine the acceptance or rejection of the null hypothesis. Under an assumption of randomness, such a statistic has a distribution of possible values. A theoretical reference distribution of this statistic under the null hypothesis is determined by mathematical methods. From this reference distribution, a critical value is determined (typically, this value is "far out" in the tails of the distribution, say out at the 99\% point). During a test, a test statistic value is computed on the data (the sequence being tested). This test statistic value is compared to the critical value. If the test statistic value exceeds the critical value, the null hypothesis for randomness is rejected. Otherwise, the null hypothesis (the randomness hypothesis) is not rejected (i.e., the hypothesis is accepted). In practice, the reason that statistical hypothesis testing works is that the reference distribution and the critical value are dependent on and generated under a tentative assumption of randomness. If the randomness assumption is, in fact, true for the data at hand, then the resulting calculated test statistic value on the data will have a very low probability (e.g., 0.01\%) of exceeding the critical value. On the other hand, if the calculated test statistic value does exceed the critical value (i.e., if the low probability event does in fact occur), then from a statistical hypothesis testing point of view, the low probability event should not occur naturally. Therefore, when the calculated test statistic value exceeds the critical value, the conclusion is made that the original assumption of randomness is suspect or faulty. In this case, statistical hypothesis testing yields the following conclusions: reject H0 (randomness) and accept Ha (non-randomness). 

Statistical hypothesis testing is a conclusion-generation procedure that has two possible outcomes, either accept H0 (the data is random) or accept Ha (the data is non-random). The following 2 by 2 table relates the true (unknown) status of the data at hand to the conclusion arrived at using the testing procedure. 



If the data is, in truth, random, then a conclusion to reject the null hypothesis (i.e., conclude that the data is non-random) will occur a small percentage of the time. This conclusion is called a Type I error. If the data is, in truth, non-random, then a conclusion to accept the null hypothesis (i.e., conclude that the data is actually random) is called a Type II error. The conclusions to accept H0 when the data is really random, and to reject H0 when the data is non-random, are correct. 

The probability of a Type I error is often called the level of significance of the test. This probability can be set prior to a test and is denoted as α. For the test, α is the probability that the test will indicate that the sequence is not random when it really is random. That is, a sequence appears to have non-random properties even when a “good” generator produced the sequence. Common values of α in cryptography are about 0.01. The probability of a Type II error is denoted as β. For the test, β is the probability that the test will indicate that the sequence is random when it is not; that is, a “bad” generator produced a sequence that appears to have random properties. Unlike α, β is not a fixed value. β can take on many different values because there are an infinite number of ways that a data stream can be non-random, and each different way yields a different β. The calculation of the Type II error β is more difficult than the calculation of α because of the many possible types of non-randomness. One of the primary goals of the following tests is to minimize the probability of a Type II error, i.e., to minimize the probability of accepting a sequence being produced by a generator as good when the generator was actually bad. The probabilities α and β are related to each other and to the size n of the tested sequence in such a way that if two of them are specified, the third value is automatically determined. Practitioners usually select a sample size n and a value for α (the probability of a Type I error – the level of significance). Then a critical point for a given statistic is selected that will produce the smallest β (the probability of a Type II error). That is, a suitable sample size is selected along with an acceptable probability of deciding that a bad generator has produced the sequence when it really is random. Then the cutoff point for acceptability is chosen such that the probability of falsely accepting a sequence as random has the smallest possible value. Each test is based on a calculated test statistic value, which is a function of the data. If the test statistic value is S and the critical value is t, then the Type I error probability is P(S > t || Ho is true) = P(reject Ho | H0 is true), and the Type II error probability is P(S ≤ t || H0 is false) = P(accept H0 | H0 is false). The test statistic is used to calculate a P-value that summarizes the strength of the evidence against the null hypothesis. For these tests, each P-value is the probability that a perfect random number generator would have produced a sequence less random than the sequence that was tested, given the kind of nonrandomness assessed by the test. If a P-value for a test is determined to be equal to 1, then the sequence appears to have perfect randomness. A P-value of zero indicates that the sequence appears to be completely non-random. A significance level (α) can be chosen for the tests. If P-value ≥ α, then the null hypothesis is accepted; i.e., the sequence appears to be random. If P-value < α, then the null hypothesis is rejected; i.e., the sequence appears to be non-random. The parameter α denotes the probability of the Type I error. Typically, α is chosen in the range [0.001, 0.01]. 

• An α of 0.001 indicates that one would expect one sequence in 1000 sequences to be rejected by the test if the sequence was random. For a P-value ≥ 0.001, a sequence would be considered to be random with a confidence of 99.9\%. For a P-value < 0.001, a sequence would be considered to be nonrandom with a confidence of 99.9\%.
• An α of 0.01 indicates that one would expect 1 sequence in 100 sequences to be rejected. A Pvalue ≥ 0.01 would mean that the sequence would be considered to be random with a confidence of 99\%. A P-value < 0.01 would mean that the conclusion was that the sequence is non-random with a confidence of 99\%.
For the examples in this document, α has been chosen to be 0.01. Note that, in many cases, the parameters in the examples do not conform to the recommended values; the examples are for illustrative purposes only. 

Considerations for Randomness, Unpredictability and Testing

The following assumptions are made with respect to random binary sequences to be tested:
1.	 Uniformity: At any point in the generation of a sequence of random or pseudorandom bits, the occurrence of a zero or one is equally likely, i.e., the probability of each is exactly 1/2. The expected number of zeros (or ones) is n/2, where n = the sequence length.
2.	 Scalability: Any test applicable to a sequence can also be applied to subsequences extracted at random. If a sequence is random, then any such extracted subsequence should also be random. Hence, any extracted subsequence should pass any test for randomness.
3.	 Consistency: The behavior of a generator must be consistent across starting values (seeds). It is inadequate to test a PRNG based on the output from a single seed, or an RNG on the basis of an output produced from a single physical output. 