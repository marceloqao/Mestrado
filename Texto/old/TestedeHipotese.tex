Seja $\bm X = (X_1, X_n)$ uma amostra a respeito da qual temos uma conjectura que queremos verificar.
Essa conjectura é a respeito dos parâmetros que caracterizam a distribuição da amostra, ou a respeito de parâmetros que caracterizam a distribuição de atributos relacionados à distribuição da amostra.
Chamaremos ``hipótese nula'' àquela que convimos em não rejeitar a não ser que obtenhamos suficiente evidência para isso; a denotaremos $H_0$.
Por vezes precisaremos da ``hipótese alternativa'', que denotaremos $H_1$.

Classicamente, um teste de hipótese se baseia em uma estatística de teste $T$ que depende exclusivamente da amostra $\bm X$, isto é, $T(\bm X)$, e é construída de tal forma que adota valores ``pequenos'' sob $H_0$ e ``cresce'' conforme ``se afasta'' de $H_0$.
Idealmente, conhecemos a distribuição de $T$ sob a hipótese nula e, com isso, somos capazes de aferir a probabilidade de observarmos valores ``grandes'' mesmo sob $H_0$.
Definimos, assim, o $p$-valor do teste baseado em $T(\bm X)$ para o valor observado $\eta$ como $\Pr_{H_0}(T(\bm X) \geq \eta)$.
O procedimento básico consiste em rejeitar a hipótese nula ao nível $100(1-\alpha)$ se o $p$-valor for inferior a $\alpha$, e em não rejeitá-la caso contrário.
Mais modernamente, não se fala em ``rejeição'', reporta-se o $p$-valor, deixando a decisão para o leitor. Desta forma, chamaremos de ``valor crítico'' o conjunto de valores, defindos pelo leitor que quando assumidos pela estatística de teste $T$ para os quais a hipótese nula $H_0$ deve ser rejeitada e, de forma análoga, utilizaremos o termo ``nível de significância'' como a probabilidade máxima de $\alpha$ acima da qual rejeita-se $H_0$




%%% MQ Escreva sobre valor crítico, nível de significância no estilo e com o vocabulário acima

