
Quão bons são os geradores de Números Aleatórios Disponíveis no R ?


No R há alguns geradores de números pseudo aleatórios bem como um pacote de testes, conhecido como Dieharder, para aferir a sua qualidade.
Há por outro lado, uma metodologia para análise de séries temporais baseada em ferramentas da teoria da informação que se mostrou útil para avaliar geradores de números pseudo aleatórios.
A proposta é avaliar os geradores disponíveis no R comparando os testes disponíveis no Dieharder com o teste baseado em ferramentas da teoria da informação.

O que são PRNG's
  Propriedades desejadas de um gerador ideal
    Um gerador de números pseudo aleatórios deve possuir algumas propriedades que garantam sua qualidade, para um leigo a construção de um gerador de números aleatórios pode parecer uma tarefa simples e alguns programadores têem demonstrado ser relativamente fácil escrever programas que gerem estranhas sequências de números aparentemente imprevisíveis. Entretanto, é bastante complexo escrever um bom programa que gere sequências satisfatórias, ou seja, uma sequência virtualmete infinita de números aleatórios estatisticamente independentes entre 0 e 1. Pois estranhas e aparentemente imprevisíveis não são necessariamente aleatórios.\cite{Sibley:88}
    Vamos parafrasear algumas afirmações feitas por dois dos diversos autores que trataram o tema.
      \textit{D. H. Lehmer (1951)}: "Uma sequência aleatória é uma vaga noção baseada na ideia de uma sequência onde cada termo é imprevisível e cujos dígitos passam em um certo número de testes, tradicionais com estatísticas ou dependendo do uso no qual a sequência será utilizada."
      \textit{J. N. Franklin (1962)}: "A sequência (1) é aleatória se possuir todas as propriedades compartilhadas por todas as infinitas sequências de amostrass independentes de variáveis aleatórias sobre a distribuição uniforme.
    A afirmação de Franklin essencialmente generaliza a de Lehmer ao dizer que a sequência precisa satisfazer \textit{todos} os testes estatísticos. Esta definição não é completamente precisa e uma interpretação sensata leva-nos a concluir que uma sequência aleatória simplesmente não existe
    
    
  Histórico: Simulação física, Von Neumann, Congruenciais Lineares, etc
    
  "Geradores disponíveis no R"
Verificação clássica das propriedades dos geradores
  Repetibilidade, portabilidade, eficiência computacional
  Testes básicos
  Testes anterior
  Testes Diehard
  Testes NIST
  Testes Dieharder
Verificação das propriedades com ferramentas da teoria da informação


  