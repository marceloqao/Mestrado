\mychapter{Introdução}{chp:introducao}
\lhead{INTRODUÇÃO}

  \lettrine{}{} 

\section{Definição do Problema}
  
%   \lettrine{N}{a} área de Redes de Sensores Sem Fios (RSSF) existem alguns problemas em aberto, ou seja, na fronteira do conhecimento, dentre estes problemas encontra-se o da Redução de Dados Multivariados em Redes de Sensores Sem Fios. Com tal estudo pretende-se aprofundar o conhecimento nas classes de problemas reais em RSSF passíveis de serem otimizados pela aplicação de técnicas de redução de dados multivariados, otimizando, desta forma, a utilização da escassa energia disponível nos nós sensores.


\section{Revisão Bibliográfica}

%   Percebe-se intensa contribuição em diversas tentativas de diminuir os dados transmitidos pelos sensores como em \emph{Collaborative data gathering in wireless sensor networks using measurement co-occurrence} \cite{Kalpakis20081979}, onde os autores oferecem um \textit{trade-off} entre o custo de comunicação versus a taxa de erros de inferência na estação base. Os resultados das simulações com dados sintéticos e reais mostraram uma redução substancial (até 65\%) nos custos de comunicação de coleta de dados para um pequeno número de erros de inferência (menos de 6\%) na estação base, já para dados reais o algoritmo provê uma economia de 27\% e um total de 1,53\% de erros de inferência. O trabalho é complementar a \cite{ISI:000233876300005}, pois manipula dados sensoriados discretos enumarados. O trabalho trata apenas o número de erros de inferência, predizer a magnitude deste erro é um dos trabalhos que o autor deixa em aberto, \cite{IdentifyingAndFilteringNearDuplicateDocuments}, utiliza \textit{min-wise} hashing para identificar a ocorrência de documentos duplicados na web estimando a sua similaridade. \cite{ISI:000233876300005}, incorporou um modelo Gaussiano Multivariado Temporal aos dados sensoriados para responder questões quando outros modelos não demonstravam a fidelidade(acurácia) desejada. Em \emph{Elliptical Anomalies in Wireless Sensor Networks} \cite{EllipticalAnomaliesinWirelessSensorNetworks} o autor trata as anomalias em RSSF que podem ocorrer devido a ataques maliciosos, sensores defeituosos, mudanças no fenômeno observado ou erros de comunicação. Definir e detectar estas anomalias em situações de limitações de energia (como é o nosso caso) é uma importante tarefa no gerenciamento deste tipo de redes. Um desafio é como detectar anomalias com baixa taxa de falsos positivos preservando assim a limitada energia da rede. Neste trabalho são definidos diferentes tipos de anomalias que ocorrem em RSSF e são providos modelos formais para os mesmos. O modelo é ilustrado utilizando-se parâmetros estatísticos em um \textit{dataset} obtido no \textit{Intel Berkeley Research Laboratory}. Os experimentos com um novo \textbf{algoritmo distribuido} de detecção de anomalias, mostram que o mesmo pode detectar anomalias elípticas com exatamente a mesma precisão (acurácia) que uma abordagem centralizada, alcançando uma redução significante de energia na rede, Finalmente é mostrado que o algoritmo é favoravelmente comparado a quatro outros métodos em quatro \textit{datasets} distintos, no trabalho duas contribuições são observadas. Primeiramente, a definição formal para anomalias, suficientemente flexível para modelar uma ampla variedade de fenômenos utilizando a geometria de \textit{Hiperelipsóides} para modelar o comportamente normal de dados sensoriados, posteriormente, foram desenvolvidos algoritmos distribuídos que analizam os dados sensoriados em suas origens, o que reduz significativamente o overhead de comunicação comparado com o modelo centralizado, que transmite todos os dados para um nó central que realiza a análise da anomalia.
%   Para ter um parâmetro de comparação dentre os diversos periódicos, utilizamos o \textit{Journal Citation Report-JCR} , dentre os periódicos consultados, destacam-se: IEEE Sensors Journal ($IF=1.581$), ACM Transactions on Sensor Networks ($IF=1.938$), Wireless Communications \& Networking ($IF=2.394$).

\section{Contribuições}

%   O trabalho proposto visa contribuir no tocante à sobrevida de Redes de Sensores sem Fios, aumentando o tempo de funcionamento das mesmas através da maior disponibilidade de energia nos nós que a constituem. O trabalho se apoia na estabilidade, desempenho e confiabilidade da linguagem R \citep{R} fazendo uso das boas práticas de \emph{Reproducible Research} como apresentadas em \citet{RR2010}.


\section{Riscos}

%   Há riscos envolvidos no processo Científico aqui tratado, dentre eles, pode-se citar

\begin{center}
	\fbox{
	\colorbox[RGB]{227, 227, 227}{
	\parbox[t]{.8\linewidth}
		{Neste capítulo tratamos de aspectos introdutórios do trabalho como a Definição do Problema, uma Revisão Bibliográfica, Contribuições e Riscos 	envolvidos no mesmo, no próximo capítulo trataremos da metodologia utilizada do desenvolvimento do mesmo.}} }
\end{center}
