\mychapter{Metodologia}{chp:metodologia}
\lhead{METODOLOGIA}

% Breve resumo do capítulo.
  \lettrine{E}{ste} capítulo tem como objetivo apresentar os materiais e métodos utilizados no trabalho. Como principal objetivo, este capítulo visa fornecer subsídios suficientes e para que o mesmo possa ser reproduzido e a continuidade do mesmo na pesquisa e desenvolvimento dos problemas deixados em aberto possa ser alcançada.

\section{Materiais e Métodos}

	Em relação a fundamentação teórica, utilizou-se como principal fonte de pesquisa a área de indexação de periódicos científicos ISI \emph{Web of Knowledge}, onde foram obtidas a grande maioria das referências, usando como parâmetros o fator de impacto dos periódicos pesquisados, a quantidade de citações de cada publicação, o grau de relevância para o tema pesquisado e o nível de produtividade (fator-H) dos autores envolvidos. O apoio em livros, surveys, lecture notes e ferramentas complementares de busca, como o \emph{google acadêmico} foram utilizadas para complementar esta pesquisa.

  Para organizar, catalogar e facilitar a consulta a todo material obtido, as referências foram gerenciadas com a ferramenta \emph{Mendeley}. Um gerenciador de referências bibliográficas multiplataforma gratuito que permite organizar de maneira centralizada vários vínculos entre as referências utilizadas, bem como visualizar e anotar as mesmas dentro da própria ferramenta. Quanto à editoração eletrônica do trabalho, fez-se uso da plataforma \LaTeX, com editor de textos \emph{Kile} de código aberto. Este trabalho foi desenvolvido num equipamento com as seguintes configurações:

\begin{center}
\begin{tabular}{c||c}
\hline
\textbf{Arquitetura} & \texttt{Intel i7 $64$ bits}\\
\hline
\textbf{S.O.} & Linux Mint 17.1 - kernel $3.13.0-43-generic$\\
\hline
\textbf{Editor} & Kile versão $2.1.3$ e \LaTeX texlive $2013.20140215-1$\\
\hline
\end{tabular}
\end{center}

  Do ponto de vista técnico deste trabalho, com ênfase em Geradores de Números Pseudoaleatórios, é utilizada a plataforma de análise estatística R . Esta plataforma foi desenvolvida originalmente por Ross Ihaka e Robert Gentleman, com o intuito de ser uma linguagem de código aberto voltada para a análise estatística e, consequentemente, a precisão numérica com fortes características funcionais \citep{R}. Por este motivo, ela será utilizada para a geração e manipulação das sequências pseudoaleatórias, análise dos dados e geração dos gráficos desse trabalho. A precisão numérica desta ferramenta, sendo aferida por \citet{Almiron2009}, é adequada para essa abordagem.

  As ferramentas utilizadas no desenvolvimento deste trabalho, são preferencialmente multiplataforma e código aberto com licença de uso \emph{GNU General Public License}~(GPL).

  Todos esses aplicativos, métodos e informações obtidas, forneceram grandes contribuições para o traçado da linha mestra deste trabalho, indicando que o mesmo está na fronteira do conhecimento produzindo um estado da arte fidedigno aos temas e ferramentas adotadas para norteá-lo.

\begin{center}
	\fbox{
	\colorbox[RGB]{227, 227, 227}{
	\parbox[t]{.8\linewidth}
		{Neste capítulo tratamos da metodologia utilizada do desenvolvimento do trabalho, no capítulo seguinte analisaremos os impactos esperados com a realização do mesmo.}} }
\end{center}
