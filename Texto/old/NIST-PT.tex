Geradores de números aleatórios (RNGs)
O primeiro tipo de gerador de sequências é um gerador de números aleatórios (RNGs). Um RNG utiliza uma fonte não-determinística, isto é, uma fonte de entropia, juntamente com alguma função de processamento, ou seja, o processo de destilação da entropia, para produzir aleatoriedade. É necessário o uso de um processo de destilação para superar qualquer falha na fonte de entropia que resulta na produção de números não aleatórios. Por exemplo, a ocorrência de longas sequências de zeros ou uns. A fonte entropia tipicamente consiste de algum fenômeno físico, como o ruído em um circuito elétrico, o escalonador de processos do usuário (por exemplo, o pressionar do teclado ou o movimentar do mouse), ou os efeitos quânticos em um semicondutor. Podem ser utilizadas várias combinações destas entradas. As saídas de um RNG podem ser usadas diretamente como um número aleatório ou podem ser a semente inicial para um gerador de números pseudo-aleatórios (PRNG). Para ser usado diretamente, ou seja, sem processamento adicional, a saída de qualquer RNG precisa satisfazer critérios rigorosos aleatoriedade como os utilizados por testes estatísticos de modo a determinar que as fontes físicas do RNG aparentem ser aleatórias. Por exemplo, uma fonte física, como o ruído eletrônico pode conter uma superposição de estruturas regulares, tais como ondas ou outros fenômenos periódicos, que podem parecer aleatórios, porém são diagnosticados não-aleatórios usando testes estatísticos. Para grande parte das utilizações a saída de um RNG precisa de ser imprevisível. No entanto, algumas fontes físicas como, por exemplo, vetores de data\/hora são bastante previsíveis. Esses problemas podem ser mitigados através da combinação de resultados a partir de diferentes tipos de fontes a serem usados como as entradas para um RNG. No entanto, as saídas resultantes dos RNG ainda podem ser deficientes quando avaliadas por testes estatísticos. Além disso, a produção de números aleatórios de alta qualidade pode ser muito demorada, tornando esta produção indesejável quando é necessária uma grande quantidade de números aleatórios. Para produzir grandes quantidades de números aleatórios, geradores de números pseudo-aleatórios podem ser preferíveis.

O segundo tipo é um gerador de números pseudo-aleatórios (PRNGs). Um PRNG usa uma ou mais entradas para gerar múltiplos números ``pseudo-aleatórios''. Dados de entrada para PRNGs são chamados de sementes. Em contextos aonde a imprevisibilidade é necessária, a semente por si só precisa ser aleatória e imprevisível. Portanto, um PRNG pode obter sua sementes da saída de RNGs, isto é, um PRNG precisa de um RNG como complemento. A saída de um PRNG é tipicamente funções determinísticas da semente, ou seja, toda a aletoriedade está contida na geração da semente. A natureza determinística do processo sugere o termo ``pseudo-aleatório''. Uma vez que cada elemento de uma sequência pseudo-aleatória é reprodutível a partir de sua semente, apenas as sementes precisam ser salvas caso haja a necessidade de validação ou reprodução dos resultados. Ironicamente, números pseudo-aleatórios

. The deterministic nature of the process leads to the term “pseudorandom.” Since each element of a pseudorandom sequence is reproducible from its seed, only the seed needs to be saved if reproduction or validation of the pseudorandom sequence is required. Ironically, pseudorandom numbers often appear to be more random than random numbers obtained from physical sources. If a pseudorandom sequence is properly constructed, each value in the sequence is produced from the previous value via transformations that appear to introduce additional randomness. A series of such transformations can eliminate statistical auto-correlations between input and output. Thus, the outputs of a PRNG may have better statistical properties and be produced faster than an RNG. 