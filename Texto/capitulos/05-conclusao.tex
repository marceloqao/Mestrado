\mychapter{Conclusão}{chp:conclusao}
\lhead{CONCLUSÃO}

%Breve resumo do capítulo.
\lettrine{A}{} análise 


\section{Trabalhos Futuros}

Uma limitação deste trabalho é que apenas verificamos a qualidade do gerador em relação a um de estrutura ideal.
Com isso, limitamos a aplicabilidade do nosso trabalho à análise de séries que, potencialmente, são ocorrências de variáveis aleatórias independentes e identicamente distribuídas.

Há farta literatura que caracteriza diferentes tipos de estruturas como, por exemplo, processos estocásticos do tipo $f^{-k}$.
A nossa metodologia pode, em princípio, ser aplicada a quaisquer processos mas, para isso, é necessário o conhecimento da distribuição dos padrões ordinais do processo de referência.
No nosso caso, trata-se da lei uniforme sobre os padrões, que é característica de ruído branco.
Não conhecemos resultados que caracterizem de forma teórica as leis de outros processos.

Há, contudo, uma solução para esse problema: estimar a lei característica do padrão de interesse.
Isso pode ser feito através de estudos Monte Carlo, mas tal extensão foge ao objetivo deste trabalho.

