\mychapter{Conclusão}{chp:conclusao}
\lhead{CONCLUSÃO}

Neste trabalho analisamos a possibilidade de a distância euclidiana de pontos no plano $(H\times C)$ de sequências ao ponto $(1,0)$, referência teórica de ruído branco, poderem ser usadas como uma estatística de teste para a hipótese de a sequência ser ruído branco.
Verificamos que essa possibilidade existe, e que essa estatística é capaz de identificar, com limitações, o mapa logístico (que já foi usado como gerador de números pseudoaleatórios), movimento browniano e ruído com autocorrelação.
Para este último, fizemos uma análise preliminar do poder do teste em função da intensidade da correlação.

Verificamos, também, que os geradores Mersenne-Twister e Randu são considerados ruído branco, mesmo sendo eles técnicas algorítmicas de geração de observações pseudoaleatórias.

%A técnica permite discriminar observações de ruído browniano (não estacionário), do mapa logístico e, com limitações, de ruído estacionário obtido pela convolução de uma sequência de ruído branco com uma máscara.
Sendo a máscara parametrizada, fizemos uma análise preliminar do poder do teste.

Uma limitação deste trabalho é que apenas verificamos a qualidade do gerador em relação a um de estrutura ideal.
Com isso, limitamos a aplicabilidade do nosso trabalho à análise de séries que, potencialmente, são ocorrências de variáveis aleatórias independentes e identicamente distribuídas.

Há farta literatura que caracteriza diferentes tipos de estruturas como, por exemplo, processos estocásticos do tipo $f^{-k}$.
A nossa metodologia pode, em princípio, ser aplicada a quaisquer processos mas, para isso, é necessário o conhecimento da distribuição dos padrões ordinais do processo de referência.
No nosso caso, trata-se da lei uniforme sobre os padrões, que é característica de ruído branco.
Não conhecemos resultados que caracterizem de forma teórica as leis de outros processos.

Há, contudo, uma solução para esse problema: estimar a lei característica do padrão de interesse.
Isso pode ser feito através de estudos Monte Carlo, mas tal extensão foge ao objetivo deste trabalho.

