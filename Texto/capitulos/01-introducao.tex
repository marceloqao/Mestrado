\mychapter{Introdução}{chp:introducao}
\lhead{INTRODUÇÃO}

  \lettrine{}{} 

\section{Motivação}
  
%   \lettrine{N}{a} .


\section{Visão geral}



%\section{Contribuições}

%   O trabalho proposto visa contribuir no tocante à sobrevida de Redes de Sensores sem Fios, aumentando o tempo de funcionamento das mesmas através da maior disponibilidade de energia nos nós que a constituem. O trabalho se apoia na estabilidade, desempenho e confiabilidade da linguagem R \citep{R} fazendo uso das boas práticas de \emph{Reproducible Research} como apresentadas em \citet{RR2010}.


%\section{Riscos}

%   Há riscos envolvidos no processo Científico aqui tratado, dentre eles, pode-se citar

\begin{center}
	\fbox{
	\colorbox[RGB]{227, 227, 227}{
	\parbox[t]{.8\linewidth}
		{Neste capítulo foram tratados os aspectos introdutórios ao tema como a motivação e uma visão geral do mesmo, no próximo capítulo é feita uma fundamentaćão teórica visando fundamentar a idéias propostas no trabalho.}} }
\end{center}

% % % Aprender Git!